\section{moebius}

\begin{lstlisting}[caption = gcd系関数]
//S(a)=g(a) - sum(i=2..a) S(floor(a/i))を求めるもの
//g(n) が O(1) レベルで求まるなら O(n^(2/3)). n<=10^11 ぐらいまでなら多分大丈夫 
map<LL,LL> results;
LL calc_func(LL n) {
 if(resultsA.find(n)!=resultsA.end())
    return resultsA[n];
  LL ret = triangle_num[n]; // g(a)
  LL last=n;
  for(LL a=2; a*a<=n; a++) ret -= calc_func(n/a),last=n/a;
  for(LL a=1; a<last; a++) ret -= calc_func(a) * (n/a-n/(a+1));
  results[n]=ret;
  return ret;
}
\end{lstlisting}

NOTE: $gcd(n,d)=1$に対する$f(d)$の数え上げについては, $n$を構成する相異なる素数$p_i$を用意して, $f(1)$から$f(d)$までの総和を$g(d)$として\\
$\sum_{p_iがb個の積でsquare-freeな積d}{ f(n)*g(n/d)(-1)^b }$のような変形ができることがある。\\
(例えば $f(n) = n^3)$のような多項式に適用できる)

\section{Monge}
重み関数$w$が任意の$a<b<c<d$で $w(a,c)+w(b,d)<w(a,d)+w(b,c)$というような具合の関数であることで、さらに$DP[left][right] = min(DP[left][k] + DP[k+1][right])+w[left][right]$を満たすことが必要である。もっと一般化すれば、次のような式に帰着させることができる。

\[
F(i,j,r) = \begin{cases}
  w(i) & i=j , p \geq 0 \\
  \min_{i \leq s <j} ( aF(i,s,f(r)) + bF(s+1,j,g(r)) +h(i,s,j) ) & f(r) \geq 0 , g(r) \geq 0
\end{cases} \]
 ただし$h(i,s,j)$はMonge関数のようなもので、次を満たすようなものである

$i \leq j \leq t < k \leq l$かつ$i \leq s < l$
\[
(t\leq s)のとき、 h(i,t,k) - h(j,t,k) + h(j,s,l) - h(i,s,l) \leq 0 , h(j,s,l) - h(i,s,l) \leq 0 \]
\[
(s \leq t)のとき、h(j,t,l) - h(j,t,k) + h(i,s,k) - h(i,s,l) \leq 0, h(i,s,k) - h(i,s,l) \leq 0 \]



例えば、一般化ハノイの塔は次の漸化式になる

\[
T(i,j,p) = \begin{cases}
w(i) & i=j , p>=0  \\
\min_{i \leq j}(2T(i,s,p) + T(s+1,j,p-1)) & i<j, p>0

\end{cases} \]
\begin{lstlisting}[caption=MongeDP]
  for(int d=1; d<n; d++) {
    for(int left=0; left+d<=n; left++) {
      int right = left+d;
      for(int k=pos[left][right-1]; k<=pos[left+1][right]; k++ ) {
	LL newNum = dp[left][k] + dp[k+1][right]
	  + (input[k+1].first - input[left].first)
	  + (input[k].second - input[right].second);
	if(newNum < dp[left][right]) {
	  dp[left][right] = newNum;
	  pos[left][right] = k;	  
	}
      }
    }
  } 
\end{lstlisting}


\section{Turanの定理}
$n$頂点をもち、部分グラフとして$K_{r+1}$を持たないとする。
このとき、辺数は高々以下の数になる
\[\frac{r-1}{r} * \frac{n*n}{2}\]

\section{Cayleyの定理}
完全グラフKnが与えられる。このとき、このグラフで作れる全域木(spanning tree)の総数は$n^{n-2}$\\

\section{リーグ戦の成績}
N人のリーグ戦があったとし、必ず勝敗が確定したものとする。\\
各人の勝ち数が与えられたとして、このような勝敗の付きかたになる勝敗の結果が存在するかの判定\\

\[存在する \Leftrightarrow \begin{cases}
0 \leq s_1 \leq ... \leq s_n <= n-1 \\
\forall i, \sum_{j=1}^{i} s_j \geq C(i,2) \\
\sum_{j=1}^{n} s_j = C(n,2)
\end{cases} \]


\section{辺の張り方}

\subsection{2-SAT}
(P or Q)のとき : addedge(not p, q) , addedge(not q, p); \\

注意 : 制約の形が元から (pならば q )だったからといって、
(not qならば not p)の制約を忘れてはならない。


\subsection{最小カット}
\[f(x) := x>0 ? 1 : 0\]
\[minimize sum(x[i] - x[j]) * c[i][j]\]
\[s.t. x[i]=0,1 x[s]=1, x[t]=0\]
とする

以下参照

$x[p]=1$ならば$x[q]=1$  $p \rightarrow q$(cap:INF)\\
$x[p]=0$ならば$x[q]=0$	$q \rightarrow p$(cap:INF)\\
$x[p]=1$ $s \rightarrow p$(cap:INF)\\
$x[p]=0$ $p \rightarrow t$(cap:INF)\\
$x[p]=1$ならば$x[q]=0$	(どちらかの$x$を反転)\\
$x[p]=0$ならば$x[q]=1$	(同上)\\

